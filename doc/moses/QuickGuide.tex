\documentclass{article}
\usepackage{url}

\newtheorem{class}{class or struct}
\newtheorem{type}{type}


\title{MOSES --- A Quick Guide}
\author{Predrag Janicic}
\author{Revised, Linas Vepstas}

\begin{document}

\maketitle

\tableofcontents

\newpage

\section{Introduction}

Meta-optimizing semantic evolutionary search (MOSES) is a new approach to
program evolution, based on representation-building and probabilistic
modeling. MOSES has been successfully applied to solve hard problems in domains
such as computational biology, sentiment evaluation, and agent control. Results
tend to be more accurate, and require less objective function evaluations, in
comparison to other program evolution systems. Best of all, the result of
running MOSES is not a large nested structure or numerical vector, but a
compact and comprehensible program written in a simple Lisp-like
mini-language. For more information see \url{http://metacog.org/doc.html}.

The list of publications on MOSES is given in the References section.
Moshe Look's PhD thesis \cite{6} should be the first material to be read
by someone interested in using or modifying MOSES.

MOSES is implemented in C++ and it heavily uses templates. So, one interested 
in modifying MOSES must be familiar with C++ and, at least to some extent, to 
C++ templates.


\section{Copyright Notice}

MOSES is Copyright 2005-2008, Moshe Looks and Novamente LLC.

It is licensed under the Apache License, Version 2.0 (the "License");
you may not use this file except in compliance with the License.
You may obtain a copy of the License at

\url{http://www.apache.org/licenses/LICENSE-2.0}

Unless required by applicable law or agreed to in writing, software
distributed under the License is distributed on an "AS IS" BASIS,
WITHOUT WARRANTIES OR CONDITIONS OF ANY KIND, either express or implied.
See the License for the specific language governing permissions and
limitations under the License.


\section{Installation}

%The code for reproducing all of the published experiments on MOSES 
%may be found in the moses/ subdirectory. A more nicely architected and
%performant implementation that is under active development may be found in the
%moses2/ subdirectory. These two directories are completely independent and use
%no common files (although identical versions of some files may be found in both
%subdirectories).

% Go into either moses/ or moses2/ (see above) and do 'make'. You will need to
% have a recent gcc (4.x or late 3.x) and the boost libraries installed
% (http://www.boost.org).

For compiling MOSES, you need 
\begin{itemize}
\item a recent gcc (4.x or late 3.x); 
\item the boost libraries (\url{http://www.boost.org});
\item the CMake package (\url{http://www.cmake.org/HTML/Index.html});
\item the ComboReduct and LADSUtil libraries (TODO : place the urls).
\end{itemize}

For compiling MOSES, create a directory \verb|bin| (from the root folder
of the MOSES distribution), go under it
and run \verb|"cmake .."|. This will create 
the needed build files. Then, make the project using "\verb|make|" (again 
from the directory \verb|bin|).
Generated executables will be in the folder \verb|bin/src/MosesEda/main|.


\section{Basic Ideas and Notions}

MOSES is an approach (and a tool) for program evolution, but it can be
used for learning different structured expressions, such as, for instance,
propositional formulas or arithmetic expressions.

Here are some of the basic notions of MOSES:

\begin{description}

\item[Scoring function.] During the learning phase, candidates being explored are scored 
by a {\em scoring function}, specific for a given problem. 
For instance, a given problem can be to learn a disjunctive 
propositional formulae over the given set of propositional 
variables, e.g., $\#1 \vee \#2 \vee \ldots \vee \#n$. 
The learning mechanism does not know this formula and instead
it only has an access to the corresponding scoring function.
A scoring function for this problem, for a given formula 
$f$ (over the variables $\#1, \#2, \ldots, \#n$) would return 
the number of rows in the truth table for $f$ such that 
the output truth value is equal to the truth value for 
$\#1 \vee \#2 \vee \ldots \vee \#n$. In this case, the 
scoring function could return values from $0$ to $2^n$. 
These values can be normalized or transformed in some other
way. Usually, the perfect score is $0$, while the worse 
scores are negative. 

\item[Exemplar and representation-building.] 
The {\em representation-building} component of MOSES constructs a 
parameterized representation of a particular region of program 
space, centered around a single program (the {\em exemplar}) or 
family of closely related programs. 

\item[Knobs.] 
A representation built (around an exemplar) is given in terms 
of {\em knobs}. A {\em knob} is a single dimension of variation 
relative to an exemplar program tree. It may be discrete or 
continuous. For example, given the program tree fragment 
$or(0<(*(\#1,0.5)),\#2)$, a continuous knob might be used to 
vary the numerical constant. So, setting this knob to $0.7$ would 
transform the tree fragment to $or(0<(*(\#1,0.7)),\#2)$. A discrete 
knob with $arity()==3$ might be used to transform the boolean input 
$\#2$. So, setting this knob to $1$ might transform the tree to 
$or(0<(*(\#1,0.7)),not(\#2))$, and setting it to $2$ might remove 
it from the tree (while setting it to $0$ would return to the 
original tree).

A program in a deme is represented as a list of knob settings
(these values can be bits, integer values, or real-number values,
depending on the knob type). The program itself can be easily 
constructed from such list and the exemplar.

\item[Deme and metapopulation.] 
A sample of programs within a representation will be referred to 
herein as a {\em deme}; a set of demes will be referred to as a 
metapopulation. MOSES operates on a metapopulation, adaptively 
creating, removing, and allocating optimization effort to 
various demes. 
\end{description}




\section{Overall Algorithm}

A basic MOSES algorithm can be described as follows:

\begin{enumerate}
\item Construct an initial set of knobs (via representation-building) based on 
some prior (e.g., based on an empty program) and use it to generate an initial 
random sampling of programs. Add this deme to the metapopulation.

\item Select a deme from the metapopulation and iteratively update its sample, as follows:

\begin{enumerate}
\item   Select some promising programs from the deme’s existing sample to use for 
        modeling, according to the scoring function. Ties in the scoring function are broken
        by preferring smaller programs.

\item   Considering the promising programs as collections of knob settings, generate
        new collections of knob settings by applying some (competent) optimization algorithm.

\item   Convert the new collections of knob settings into their corresponding programs,
        evaluate their scores, and integrate them into the deme’s sample, replacing less
        promising programs.
\end{enumerate}

\item  For each of the new programs that meet the criteria for creating a new deme, if any:

\begin{enumerate}
\item   Construct a new set of knobs (via representation-building) to define a region
        centered around the program (the deme’s exemplar), and use it to generate a
        new random sampling of programs, producing a new deme.

\item   Integrate the new deme into the metapopulation, possibly displacing less 
        promising demes.
\end{enumerate}

\item Repeat from step 2.
\end{enumerate}

Note that representation-building and optimization algorithm that parametrize
the above algorithm are vital components of the system and they crucially 
influence its performance.
Representation building is specific for each domain (e.g., learning 
propositional formulae), while optimization algorithm is general. 
Currently, in MOSES, there is support for representation building 
for several problem domains (e.g., propositional formulae, 
actions\footnote{By ``actions'' we mean mini programming languages
describing actions of a agents such as artificial and \cite{6}. Available
actions typically cover atomic instructions like ``step forward'',
``rotate left'', ``rotate right'', ``step forward'', 
branching instruction such as ,,if-then-else``,
and loop instruction such as ,,while``.}) 
and several optimization algorithms. 


\section{Source Files and Folders}

In the MOSES distribution, there are the following folders (under
\verb|moses/learning/moses|) with the source files:

\begin{description}
\item[moses] this is the folder with the core support for MOSES --- 
support for representation building, some scoring functions, optimization
algorithms, etc. 

\item[main] this is the folder with files giving examples of executables 
demonstrating different features of the MOSES system: representation building, 
reducing expressions, and, of course, applications of MOSES itself (for 
instance, there are examples for the "ant problem", for "parity formulae", etc.

\item[eda] support for estimation of distribution algorithms with lower
level support for optimization algorithms. 
\end{description}

%MOSES is a part of a bigger project, so there are some functions (or even files)
%that are not used by MOSES (but are used elsewhere).

%In addition, there are the folders with \verb|scripts|, \verb|sample|, and 
%\verb|test|.


\section{Key Types, Structures, and Classes}

This section briefly discusses technical details about the key elements of MOSES.

\subsection{Structured Expressions}

For representing structured expressions (programs, propositional formulae, etc)
MOSES relies on the library ComboReduct.

In ComboReduct, structured expressions are represented 
by trees of the type \verb|combo_tree| as follows:

\verb|typedef Util::tree<vertex> combo_tree;|

The file \verb|comboreduct/combo/vertex.h| defines
\verb|vertex| as shown blow.  It is done this way so that it 
can capture different sorts of nodes, for different, but still
fixed, problem domains.

\begin{verbatim}
  typedef boost::variant<builtin,
                         wild_card,
                         argument,
                         contin_t,
                         action,
                         builtin_action,
                         perception,
                         definite_object,
                         indefinite_object,
                         message,
                         procedure_call,
                         ann_type,
                         action_symbol> vertex;
\end{verbatim}

For a detailed explanation, review the docs provided 
in the distribution of the ComboReduct library.

\subsection{Metapopulation}

The metapopulation is a set of scored combo trees. Several types of scores 
are used, ranging from single floats, to composite scores that include 
a complexity measure, to 'behavioural' scores, consisting of vectors of floats.
These are all defined in the file
\verb|learning/moses/moses/types.h|

A 'scored combo tree' is then just a pair that associates a score with a tree:

\begin{verbatim}
typedef tagged_item<combo::combo_tree,
                    composite_behavioral_score> bscored_combo_tree;
\end{verbatim}

The metapopulation is then a set of scored combo trees, defined in 
\verb|learning/moses/moses/metapopulation.h| (more precisely, it is a
class, inheriting from the set):

\begin{verbatim}
  typedef std::set<bscored_combo_tree,
                   bscored_combo_tree_greater> bscored_combo_tree_set;

  struct metapopulation : public bscored_combo_tree_set {...};
\end{verbatim}

The metapopulation will store expressions (as scored trees) that were
encountered during the learning process (not all of them; the weak
ones, which are dominated by existing ones, are usually skipped as
non-promising). 

As an example, one can iterate through the metapopulation and print 
all its elements with their scores and complexities in the following way:

\begin{verbatim}
 for (const_iterator it=begin(); it!=end(); ++it)
      cout << get_tree(*it) << " " 
           << get_score(*it) << " " 
           << get_complexity(*it) << endl;
\end{verbatim}

The metapopulation is updated in iterations. In each iteration, one 
of its elements is selected as an exemplar. The exemplar is then
used for building a new deme (that will further extend the
metapopulation).


\subsection{Deme}

The metapopulation consists of expressions (stored as scored trees).
These expressions do not refer to some exemplar. On the other hand,
during the learning process, new demes are generated. 
A deme is centered around an exemplar from the metapopulation. 
However, elements of the deme are not represented as (scored) trees
but, instead, are represented relative to the exemplar. 
So, the deme has the type (declared in \verb|eda/instance_set.h|): 

\begin{verbatim}
eda::instance_set<tree_score>
\end{verbatim}

\noindent
where \verb|instance_set| is a set of scored instances, and \verb|instance|
is declared in \verb|eda/eda.h| is a vector of packed knob settings: 

\begin{verbatim}
typedef vector<packed_t> instance;
\end{verbatim}

The basic learning and optimization processes work over these instances,
and not over trees. So, a suitable way for representing expressions/trees 
as sequences of bit/integer/real values (relative to the exemplar) enable
abstract and clean optimization steps, uniform for different domains.
This representation is based on representation building relative to a
given exemplar.


\subsection{Representation}

A structure describing a representation of a deme (relative to its
exemplar) is declared 
in \verb|moses/representation.h|. Its constructor uses as arguments
a simplification rule, an exemplar, a type tree, a pseudo-number
generator and, optionally, just for actions (like for the ant problem),
sets of available operators, perceptions, and actions).

The constructor of this structure, builds knobs with respect to the
given exemplar (by the method \verb|build_knobs|).

This structure stores the exemplar like a tree (more precisely \verb|vtree|).
This structure has a method for using a given instance to transform 
the exemplar (\verb|transform|) providing a new expression tree. 

The structure also has methods for clearing the current version of
the exemplar (setting all knobs to default values --- zeros) ---
\verb|clear_exemplar|, for getting the exemplar ---
\verb|get_clean_exemplar|, and for getting the reduced, simplified 
version of the exemplar \verb|get_clean_exemplar|.


\section{Key Methods}

This section briefly discusses key MOSES's methods.

\subsection{Representation Building}

Representation building is one of the key aspects of the MOSES approach. 
It is implemented separately for different problem domains (propositional
formulae, action, etc). Support for several problem domains is given in 
the file \verb|moses/build_knobs.h/cc|. 
Representation building starts with an exemplar and add to it new nodes
and corresponding knobs (see \cite{6}). Knobs can have different settings. 
If all knobs are set to 0, then we the original exemplar is obtained. 

There are several types of knobs, described in \verb|moses/knobs.h|.
Some of them are suitable for propositional formulae (so some subexpression
can be {\em present}, {\em absent}, or {\em negated}. In simple knobs,
a subexpression can be just {\em present} or {\em absent}.
In action knobs, a node can have different settings, corresponding
to atomic or compound actions, sampled as ``perms'' 
in the method \verb|build_knobs::sample_action_perms|.

The key method is \verb|void build_knobs::build_action(pre_it it)|
--- it substantially determines the representation building for
one domain and substantially influence the learning process.

\subsection{Optimization}

The role of optimization is to score elements of the deme, and to 
process them and to generate new promising instances. The optimization 
process works over sequences of numbers (i.e., over ``instances'').
It is invoked in the main procedure, in each iteration of expanding
(the method \verb|metapopulation::expand|, implemented in the file
\verb|moses/moses.h|) the metapopulation. The result is a set of 
instances that are transformed to trees and then added to the 
metapopulation (if they are not dominated by existing elements).  

Currently, there are two optimization methods implemented
(implemented in the file \verb|moses/optimization.h|): 
\begin{description}
\item[univariate optimization] based on Bayesian optimization (see \cite{6});

\item[iterative hillclimbing] based on a simple greedy iterative process, 
looking for a better than exemplar instance at distance $1, 2, 3, \ldots$;
\end{description}


In addition, there is support for ``sliced iterative hillclimbing'',
similar as the one as above, but using a time slicing, so it can be used
 in some wider context, without leaving other subtasks idle for too long.

Optimization algorithms also have to take care of number of evaluations
(number of calls to scoring functions) used. Basically, this number
controls the resources given to the algorithm.


\subsection{Scoring}

While the representation building is specific for each problem domain,
the scoring is specific for each specific problem. For instance, in
learning propositional formulae, the same representation building
algorithm is used, but different scoring functions will be used
for each specific task (for instance, for learning disjunction
or conjunction over the given set of propositional variables).

There is a 
\begin{description}
\item[score function] returning \verb|int|, generally with
0 as a perfect score, and negative numbers as worse scores;
\item[behavioral score function] returning a vector of specific 
values, that is used for comparing expressions on different 
dimensions and for discarding elements dominated by other elements.
\end{description}

Technically, these functions are provided as operators within 
structures. 

The scoring functions are used for instantiating higher-level
functions used uniformly for different problem domains and different
problems. 

Some low-level, problem specific, scoring functions are defined in 
the file (under \verb|src/MosesEda|) \verb|moses/scoring_functions.h|,
while higher-level support is defined in the file \verb|moses/scoring.h/cc|.

\section{MOSES: Putting It All Together}

With all components briefly described above, this section discusses
how are they combined in a system MOSES. 

The main moses method is trivial: it only expand the metapopulation 
in iterations until the given number of evaluations or a perfect
solution is reached. This method is implemented in \verb|moses/moses.h|,
in several variations (some with additional arguments corresponding to
available actions and perceptions, just for the action problem domain). 

Typical usage of MOSES starts by providing scoring functions. For 
instance, for learning disjunction propositional formula one can 
use the following declaration (defined in \verb|moses/scoring_functions.h|):

\begin{verbatim}
disjunction scorer;
\end{verbatim}

\noindent 
and for solving the ant problem, one can use the following declaration 
(defined in \verb|moses/scoring_functions.h|):

\begin{verbatim}
ant_score scorer;
\end{verbatim}

Also, the type of expression to be learnt has to be provided 
\footnote{for a detail explanation of the type system used in ComboReduct
  see the doc provided with the distribution of ComboReduct}.
For instance, for the disjunctive formula, one should use:

\begin{verbatim}
type_tree tt(id::lambda_type);
tt.append_children(tt.begin(),id::boolean_type,arity+1);
\end{verbatim}

\noindent 
where \verb|arity| carries the information of the number of 
propositional variables to be considered. For the ant problem, 
one would write: 

\begin{verbatim}
type_tree tt(id::lambda_type);
tt.append_children(tt.begin(),id::action_result_type,1);
\end{verbatim}

Then the metapopulation has to be declared. It is instantiated
via templates, saying which scoring function, which behavioral scoring
function, and which optimization algorithm to use. As, arguments
one has to provide the random generator, the initial exemplar,
the type tree, simplification procedure, then the scorers and
the optimization algorithm. This is an example for learning
the disjunctive formula:

\begin{verbatim}
metapopulation<logical_score,logical_bscore,univariate_optimization> 
  metapop(rng, vtree(id::logical_and),tt,logical_reduction(),
          logical_score(scorer,arity,rng),
          logical_bscore(scorer,arity,rng),
          univariate_optimization(rng));
\end{verbatim}

\noindent 
and this is an example for the ant problem: 

\begin{verbatim}
metapopulation<ant_score,ant_bscore,univariate_optimization> 
  metapop(rng,vtree(id::sequential_and),tt,action_reduction(),
          scorer,
          bscorer,
          univariate_optimization(rng));
\end{verbatim}

There is also a version of MOSES that uses the sliced interactive
hillclimbing --- \verb|sliced_moses|. It supports the action domain,
but can be simply modify to support other domains as well.

These, and several more examples, can be found in the folder \verb|main|.
The example programs provided often ask for arguments like the seed for
the pseudo-random number generation or for the number of evaluations.



\section{Final Remarks}

MOSES is a rather big system and it cannot be documented in details
in a few pages. However, the descriptions given above should be 
helpful when one first encounters MOSES and tries to use it and 
modify it. 



\begin{thebibliography}{10}

\bibitem{1}
Moshe Looks, "Scalable Estimation-of-Distribution Program Evolution",
Genetic and Evolutionary Computation COnference (GECCO), 2007.

\bibitem{2} 
Moshe Looks, "On the Behavioral Diversity of Random Programs", Genetic and
Evolutionary Computation COnference (GECCO), 2007.

\bibitem{3} 
Moshe Looks, "Meta-Optimizing Semantic Evolutionary Search", Genetic and
Evolutionary Computation COnference (GECCO), 2007.

\bibitem{4}
Moshe Looks, Ben Goertzel, Lucio de Souza Coelho, Mauricio Mudado, and
Cassio Pennachin,"Clustering Gene Expression Data via Mining Ensembles of
Classification Rules Evolved Using MOSES", Genetic and Evolutionary
Computation COnference (GECCO), 2007.

\bibitem{5} 
Moshe Looks, Ben Goertzel, Lucio de Souza Coelho, Mauricio Mudado, and
Cassio Pennachin, "Understanding Microarray Data through Applying Competent
Program Evolution", Genetic and Evolutionary Computation COnference (GECCO),
2007.

\bibitem{6} 
Moshe Looks, "Competent Program Evolution" Doctoral Dissertation, Washington
University in St. Louis, 2006.

\bibitem{7} 
Moshe Looks, "Program Evolution for General Intelligence", Artificial
General Intelligence Research Institute Workshop (AGIRI), 2006.
\end{thebibliography}


\end{document}











